\documentclass[12pt]{article}

\usepackage{amsmath}
\usepackage{amsfonts}
\usepackage{float}
\usepackage{fancyhdr}
\usepackage{graphicx}
\usepackage[colorlinks=true,linkcolor=blue, citecolor=red]{hyperref}
\usepackage{url}
\usepackage[top=.75in, left=.5in, right=.5in, bottom=1in]{geometry}
\usepackage[utf8]{vietnam}
\setlength{\headheight}{29.43912pt}

\pagestyle{fancy}
\lhead{
Báo cáo Lab 02 - Naive Bayes Classifier
}
\rhead{
Trường Đại học Khoa học Tự nhiên - ĐHQG HCM\\
CSC150008 - Xử lý ngôn ngữ tự nhiên ứng dụng
}
\lfoot{\LaTeX\ by \href{https://github.com/trhgquan}{Quan, Tran Hoang}}

\begin{document}
\noindent Sinh viên thực hiện: Trần Hoàng Quân (MSSV: 19120338)
\section{Chạy chương trình}
File thực thi là \texttt{main.py}. Thầy chạy \texttt{pip install -r requirements.txt} trước, sau đó chạy \texttt{python main.py} là được.
\section{Tiền xử lí \& huấn luyện mô hình}
Dữ liệu văn bản trải qua các bước tiền xử lí sau:
\begin{itemize}
\item Chuyển thành chữ thường.
\item Loại bỏ dấu câu.
\item Loại bỏ chữ số có trong câu.
\end{itemize}
Vì có sự chênh lệch lớn giữa số lượng câu có nhãn 2 với các nhãn còn lại, nên em chỉ chọn mỗi nhãn 86 câu để huấn luyện (bằng với số câu của nhãn 2).

\section{Kết quả huấn luyện}
Test trên 500 câu trong tập tin test cho kết quả như sau:
\begin{table}[H]
\centering
\begin{tabular}{|c|c|c|c|}
\hline
label & P (precision) & R (recall) & f1-score \\
\hline
0 & 0.67 & 0.89 & 0.77 \\
1 & 0.50 & 0.44 & 0.47 \\
2 & 0.33 & 0.33 & 0.33 \\
3 & 0.40 & 0.43 & 0.41 \\
4 & 0.46 & 0.47 & 0.46 \\
5 & 0.64 & 0.42 & 0.51 \\
\hline
Tổng & 0.56 & 0.56 & 0.55 \\
\hline
\end{tabular}
\caption{Kết quả huấn luyện}
\end{table}

\section{Phát triển thêm - \today}
Phần này phát triển thêm so với phần đã làm bên trên. Ở đây mình cài bằng thuật toán mô tả trong sách Speech and Language Processing (3rd edition draft) của thầy Dan Jurafsky\cite{10.5555/555733}. Khác biệt lớn nhất ở phần này mình sẽ \textbf{không bỏ qua stopwords}. Kết quả huấn luyện \& testing\footnote{với training \& testing dataset như ban đầu} ở bảng sau:
\begin{table}[H]
\centering
\begin{tabular}{|c|c|c|c|}
\hline
label & P (precision) & R (recall) & f1-score \\
\hline
0 & 0.83 & 0.75 & 0.79 \\
1 & 0.62 & 0.61 & 0.61 \\
2 & 0.88 & 0.78 & 0.82 \\
3 & 0.81 & 0.95 & 0.87 \\
4 & 0.68 & 0.94 & 0.79 \\
5 & 0.97 & 0.73 & 0.83 \\
\hline
Accuracy &  &  & 0.55 \\
Macro AVG & 0.80 & 0.79 & 0.79 \\
Weighted AVG & 0.79 & 0.78 & 0.78 \\
\hline
\end{tabular}
\caption{Kết quả huấn luyện lúc sau}
\end{table}



\cleardoublepage
\phantomsection
\addcontentsline{toc}{section}{Tài liệu}
\bibliographystyle{plain}
\bibliography{main}
\end{document}